\section{引言}
这是一个适用于西南财经大学本科生毕业论文规范的 \LaTeX \cite{lamport1994latex,goossens1994latex}模板.
\TeX 是一个由美国计算机教授\citet{knuth1984texbook}编写的排版软件。
你可以根据你自己的需要增删章节\footnote{这是一段脚注}. 下面是一些例子.

\subsection{公式}
\autoref{eq:eq1} 是一个著名的等式.
\begin{equation}\label{eq:eq1}
    \frac{\pi^2}{6}=\sum_{n=1}^{\infty}\frac{1}{n^2}
\end{equation}
\subsection{表格}
简单表格可以直接输入,复杂表格可以使用 \href{https://github.com/krlmlr/Excel2LaTeX}{Excel2LaTeX}将 Excel 表格转换成 \LaTeX 代码,或者使用 \href{https://www.tablesgenerator.com/latex_tables}{LaTeX Tables Generator
},然后复制粘贴.\autoref{tab:tab1}是一个例子。 
% Table generated by Excel2LaTeX from sheet 'Sheet1'
\begin{table}[htbp]
    \centering
    \caption{计数统计}
      \begin{tabular}{|l|rrr|rrr|rrr|rrr|}
      \hline
        & \multicolumn{3}{c|}{model1} & \multicolumn{3}{c|}{model2} & \multicolumn{3}{c|}{model3} & \multicolumn{3}{c|}{model4} \bigstrut\\
  \cline{2-13}    label & 0 & 1 & \multicolumn{1}{l|}{总计} & 0 & 1 & \multicolumn{1}{l|}{总计} & 0 & 1 & \multicolumn{1}{l|}{总计} & 0 & 1 & \multicolumn{1}{l|}{总计} \bigstrut\\
      \hline
      \multicolumn{1}{|r|}{0} & 9659 & 1206 & 10865 & 9659 & 1206 & 10865 & 9659 & 1206 & 10865 & 9659 & 1206 & 10865 \bigstrut[t]\\
      \multicolumn{1}{|r|}{1} & 1206 & 493 & 1699 & 1206 & 493 & 1699 & 1206 & 493 & 1699 & 1206 & 493 & 1699 \\
      总计 & 10865 & 1699 & 12564 & 10865 & 1699 & 12564 & 10865 & 1699 & 12564 & 10865 & 1699 & 12564 \bigstrut[b]\\
      \hline
      \end{tabular}%
    \label{tab:tab1}%
  \end{table}%
  
  \subsection{图片}
  \autoref{fig:fig1}是一个示例.

  \begin{figure}[htbp]
      \centering
      \includegraphics{swufe-Logo/opaque.png}
      \caption{西南财经大学Logo}
      \label{fig:fig1}
  \end{figure}
  \subsection{教程}
  \LaTeX 教程建议参考《一份(不太)简短的 \LaTeXe 介绍》. 使用命令\verb"texdoc lshort-zh" 快速打开。下面介绍模板定义的一些环境与其他不太常用的命令。  
  模板定义了中英文摘要环境。
  \begin{lstlisting}[language=TeX]
    \begin{abstractzh}
      这是中文摘要
      \keywordszh{中文关键词;中文关键词}
    \end{abstractzh}
    \begin{abstracten}
      Abstract in English
      \keywordsen{Keyword1, keyword2}
    \end{abstracten}
  \end{lstlisting}
Hyperref 包中的 \verb"\autoref" 命令是一个很方便的交叉引用的命令。使用 \verb"\autoref{fig:fig1}"可以自动产生\autoref{fig:fig1}. 这是根据图表、公式、章节的标签来自动处理的。 对于图片,建议的标签格式为 \verb"fig:...", 公式为 \verb"eq:...", 表格为 \verb"tab:...", 节为 \verb"sec:..."。

模板还根据完整版和检测版进行了不同的处理,默认是检测版,去除了封面、目录、版权申明、封底。 你需要手动注释掉致谢、后记和附录。
\begin{lstlisting}[language=TeX]
  \documentclass{swufethss-b} % 默认检测版
  \documentclass[final]{swufethss-b} % 完整版
\end{lstlisting}